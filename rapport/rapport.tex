%%%%%%%%%%%%%%%%%%%%%%%%%%%%%%%%%%%%%%%%%
% University Assignment Title Page
% LaTeX Template
% Version 1.0 (27/12/12)
%
% This template has been downloaded from:
% http://www.LaTeXTemplates.com
%
% Original author:
% WikiBooks (http://en.wikibooks.org/wiki/LaTeX/Title_Creation)
%
% License:
% CC BY-NC-SA 3.0 (http://creativecommons.org/licenses/by-nc-sa/3.0/)
%
% Instructions for using this template:
% This title page is capable of being compiled as is. This is not useful for
% including it in another document. To do this, you have two options:
%
% 1) Copy/paste everything between \begin{document} and \end{document}
% starting at \begin{titlepage} and paste this into another LaTeX file where you
% want your title page.
% OR
% 2) Remove everything outside the \begin{titlepage} and \end{titlepage} and
% move this file to the same directory as the LaTeX file you wish to add it to.
% Then add \input{./title_page_1.tex} to your LaTeX file where you want your
% title page.
%
%%%%%%%%%%%%%%%%%%%%%%%%%%%%%%%%%%%%%%%%%
%\title{Title page with logo}
%----------------------------------------------------------------------------------------
%   PACKAGES AND OTHER DOCUMENT CONFIGURATIONS
%----------------------------------------------------------------------------------------

\documentclass[12pt]{article}
\usepackage[utf8x]{inputenc}
\usepackage[T1]{fontenc}
\usepackage[francais]{babel}
\usepackage{amsmath}
\usepackage{graphicx}
\usepackage[colorinlistoftodos]{todonotes}
\usepackage{minted}

\setlength{\parindent}{0cm} % indentation a 0 de base

\begin{document}

\begin{titlepage}

\newcommand{\HRule}{\rule{\linewidth}{0.5mm}} % Defines a new command for the horizontal lines, change thickness here

\center % Center everything on the page

%----------------------------------------------------------------------------------------
%   HEADING SECTIONS
%----------------------------------------------------------------------------------------

\textsc{\LARGE }\\[1.5cm] % Name of your university/college
\textsc{\Large Université de Nantes}\\[0.5cm] % Major heading such as course name
\textsc{\large IUT de Nantes}\\[0.5cm] % Minor heading such as course title

%----------------------------------------------------------------------------------------
%   TITLE SECTION
%----------------------------------------------------------------------------------------

\HRule \\[0.4cm]
{ \huge \bfseries Codes Correcteurs}\\[0.4cm] % Title of your document
\HRule \\[1.5cm]

%----------------------------------------------------------------------------------------
%   AUTHOR SECTION
%----------------------------------------------------------------------------------------

\begin{minipage}{0.4\textwidth}
\begin{flushleft} \large
\emph{Auteurs:}\\
Brewal \textsc{Henaff}\\
Cédric \textsc{Berland}\\
Nathan \textsc{Maraval}
\end{flushleft}
\end{minipage}
~
\begin{minipage}{0.4\textwidth}
\begin{flushright} \large
\emph{Cours:} \\
Modélisation Mathémathique % Supervisor's Name
\end{flushright}
\end{minipage}\\[3cm]


%----------------------------------------------------------------------------------------
%   DATE SECTION
%----------------------------------------------------------------------------------------

{\large \today}\\[2cm] % Date, change the \today to a set date if you want to be precise

%----------------------------------------------------------------------------------------
%   LOGO SECTION
%----------------------------------------------------------------------------------------


\includegraphics[width=4.7cm]{logo.jpg}\\%[1cm] % Include a department/university logo - this will require the graphicx package

%----------------------------------------------------------------------------------------

\newpage % Fill the rest of the page with whitespace

\end{titlepage}

%----------------------------------------------------------------------------------------
%   TABLE OF CONTENTS
%----------------------------------------------------------------------------------------
\renewcommand{\contentsname}{Sommaire}
\tableofcontents
\newpage

%----------------------------------------------------------------------------------------
%   INTRODUCTION
%----------------------------------------------------------------------------------------
\section{Introduction}
\label{sec:introduction}

Ce projet à été réalisé dans le cadre de la formation en Modélisation Mathematique, en DUT Informatique à l'IUT de Nantes.

Il consiste a concevoir un programme permettant l'encodage d'un message binaire, puis par différentes methode que nous expliciterons plus tard, de le décoder et de corriger les possibles erreurs de transmission.

%----------------------------------------------------------------------------------------
%   CONTENTS
%----------------------------------------------------------------------------------------

\section{La théorie}
\label{sec:theorie}

Cette partie couvrira la théorie, les opérations effectués lors de ce projet, que ce soit lors de la transmition, ou bien de la récéption, comme du brouillage qui sera effectué pour pouvoir tester les données.

\subsection{La détection d'erreur}
\label{sub:La détection d'erreur}

Il existe plusieurs méthodes pour détecter les erreurs, par exemple associer un mot à une lettre comme utilisé dans l'armée :

\hspace{1cm} exemple : erreur $\rightarrow$ Echo Romeo Romeo Echo Uniforme Romeo
\\
\\ En informatique on utilise ce qu'on appel des ``bits de parité'' qui indique indique si le nombre de 1 dans l'octet est pair ou non

\hspace{1cm} exemple : A $\rightarrow$ 1000001 et ajoutera donc un 0 puisse que il y a deux 1 on obtiendra donc pour A le code suivant ``01000001''.
\\ Le bits de parité nous permet de savoir si lors de la transmition le message à été modifié, simplement en regardant si le bit de parité correspond toujours au reste du message (si le nombre de 1 est toujours pair).

\subsection{La correction d'erreur}
\label{sub:La correction d'erreur}

Les différents exemple ci-dessus ne permettent que de détecter les erreurs. Mais ce qui nous intéresse vraiment c'est la correction des erreurs.
\\ Il existe des moyens simple, par exemple la duplication du message :

\hspace{1cm} ``erreur'' $\rightarrow$ eee rrr rrr eee uuu rrr
\\ On répète chaque lettre 3 fois. Pourquoi 3 fois ? Et pas 2 ?
\\ Imaginons que l'on multiplie seulement 2 fois, on reçois le message suivant : ``ââgmee'', le message de base peut être ``âge'' mais aussi ``âme''.
\\ En revanche si l'on multiplie 3 fois on recevra donc un message tel que ``âââgmmeee'', en supposant qu'il n'y ai eu qu'une erreur, le message envoyer est donc ``âme''.

\hspace{1cm} Le problème de ce genre de code de correction c'est qu'ils sont très lourd, et coûteux, on transmet trois fois les données, et lorsque l'on voit le poids des données généralement transmissent (images, musiques, vidéos), on s'en rend vite compte.

\subsection{Le code de Hamming}
\label{sub:Le code de Hamming}

Le code de Hamming est un code correcteur moderne.
\\ Généralement on utilise un code de Hamming C(7,4). C'est à dire que l'on envoie 7 bits et que les 4 premiers contiennent les données à transmettre. Les 3 derniers servent à la détection des erreurs.
\\ Ces 3 derniers bits sont en fait l'addition, bit à bit, des 4 autres. En pratique cela donne ça :

\hspace{1cm} Soit $v_1, v_2, \ldots, v_7$ les bits transmit et $u_1, u_2, u_3$ et $u_4$ les bits contenant le message.
\begin{align*}
  v_1 &= u_1,\\
  v_2 &= u_2,\\
  v_3 &= u_3,\\
  v_4 &= u_4,\\
  v_5 &= u_1 + u_2 + u_4,\\
  v_6 &= u_1 + u_3 + u_4,\\
  v_7 &= u_2 + u_3 + u_4,
\end{align*}

Lorsque l'on reçoit le message (bits $w_1, w_2, \ldots, w_7$), on est face à 8 possibilités :

  (0) il n'y a pas d'erreur; \\
  (1) $w_1$ est erronée; \\
  (2) $w_2$ est erronée; \\
  (3) $w_3$ est erronée; \\
  (4) $w_4$ est erronée; \\
  (5) $w_5$ est erronée; \\
  (6) $w_6$ est erronée; \\
  (7) $w_7$ est erronée.

Grâce aux 3 derniers bits on peut savoir d'où provient l'erreur. Il suffit de les recalculer (bits $W_5, W_6, W_7$) et de les comparer, pour obtenir un des huit cas précédent :

  (0) si $w_5 = W_5$ et $w_6 = W_6$ et $w_7 = W_7$;\\
  (1) si $w_5 \ne W_5$ et $w_6 \ne W_6$; \\
  (2) si $w_5 \ne W_5$ et $w_7 \ne W_7$; \\
  (3) si $w_6 \ne W_6$ et $w_7 \ne W_7$; \\
  (4) si $w_5 \ne W_5$ et $w_6 \ne W_6$ et $w_7 \ne W_7$; \\
  (5) si $w_5 \ne W_5$; \\
  (6) si $w_6 \ne W_6$; \\
  (7) si $w_7 \ne W_7$. \\

Cependant ce code ne permet de détecter et corriger qu'une seule erreur.

\section{Notre code}
\label{sec:Notre code}

\subsection{L'envoi du message}
\label{sub:L'envoi du message}

Test de code :

\begin{minted}{c}
int GMatrix[4][8] = {{1,1,0,1,1,0,0,0},
                     {0,1,1,1,0,1,0,0},
                     {1,0,1,1,0,0,1,0},
                     {1,1,1,0,0,0,0,1}} ;
char G[4] ;
void init_generators() {
 for (int i=0; i < 4; ++i) {
   G[i] = 0 ;
   for (int j=0; j < 8; ++j)
     if (GMatrix[i][j]) G[i] |= (1 << (7-j)) ;
 }
}

 void hamming(char c, char out[2]) {
   out[0] = out[1] = 0;
   for (int i=0; i < 4; ++i) {
     if (c & (1 << i))
       out[1] ^= G[i] ;
     if (c & (1 << (i+4)))
       out[0] ^= G[i] ;
   }
 }


\end{minted}



\subsection{Le brouillage}
\label{sub:brouillage}

Afin de pouvoir tester notre code dans des conditions d'erreurs réelles, nous utilisons un programme permettant de brouiller certains bits, aléatoirements. Ce brouillage nous permet de tester notre code décodeur, étant donné que la probabilité d'un brouillage dû au code d'encodage est extremement faible.

Pour se faire, nous utilisons le code transmit.c, qui simule une transmition de message (comme par example l'envois d'un message à un satelite en orbite, ce qui peut engendrer de nombreuses erreurs).



\subsection{La réception et le décodage du message}
\label{sub:La réception et le décodage du message}



\section{d'assault}
\label{sec:assault}


Comments can be added to the margins of the document using the \todo{Here's a comment in the margin!} todo command, as shown in the example on the right. You can also add inline comments too:

\todo[inline, color=green!40]{This is an inline comment.}

\subsection{Tables and Figures}

Use the table and tabular commands for basic tables --- see Table~\ref{tab:widgets}, for example. You can upload a figure (JPEG, PNG or PDF) using the files menu. To include it in your document, use the includegraphics command as in the code for Figure~\ref{fig:frog} below.

% Commands to include a figure:
%\begin{figure}
%\centering
%\includegraphics[width=0.5\textwidth]{frog.jpg}
%\caption{\label{fig:frog}This is a figure caption.}
%\end{figure}

%\begin{table}
%\centering
%\begin{tabular}{l|r}
%Item & Quantity \\\hline
%Widgets & 42 \\
%Gadgets & 13
%\end{tabular}
%\caption{\label{tab:widgets}An example table.}
%\end{table}

\subsection{Mathematics}

\LaTeX{} is great at typesetting mathematics. Let $X_1, X_2, \ldots, X_n$ be a sequence of independent and identically distributed random variables with $\text{E}[X_i] = \mu$ and $\text{Var}[X_i] = \sigma^2 < \infty$, and let
$$S_n = \frac{X_1 + X_2 + \cdots + X_n}{n}
      = \frac{1}{n}\sum_{i}^{n} X_i$$
denote their mean. Then as $n$ approaches infinity, the random variables $\sqrt{n}(S_n - \mu)$ converge in distribution to a normal $\mathcal{N}(0, \sigma^2)$.

\subsection{Lists}

You can make lists with automatic numbering \dots

\begin{enumerate}
\item Like this,
\item and like this.
\end{enumerate}
\dots or bullet points \dots
\begin{itemize}
\item Like this,
\item and like this.
\end{itemize}

We hope you find write\LaTeX\ useful, and please let us know if you have any feedback using the help menu above.

%----------------------------------------------------------------------------------------
%   ANNEXES SECTIONS
%----------------------------------------------------------------------------------------
\appendix
\section{Annexe 1}
\label{sec:Annexe 1}


\end{document}
